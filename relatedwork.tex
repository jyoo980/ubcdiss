%% The following is a directive for TeXShop to indicate the main file
%%!TEX root = diss.tex

\chapter{Related Work}
\label{ch:RelatedWork}

\begin{epigraph}
    \emph{
       The computing scientist’s main challenge is not to get confused by the 
       complexities of his own making.
     } ---~Edsger W. Dijkstra
\end{epigraph}

\noindent Researchers have investigated both the questions that developers ask 
as they understand and evolve their systems, as well as the efficacy of tools 
that support developers in answering their questions during software change 
tasks.

\par Developers find it challenging to understand code in modern codebases for
a number of reasons.
Many codebases today are large and complex, with some meeting the definition of 
an \emph{ultra-large-scale system} \cite{feiler-2006-ulss}.

\par Google reported in 2016 that its monolithic software codebase was composed
of approximately 1 billion files, with a history of 35 million commits in a
repository that contains 85 terabytes of data and 2 billion lines of code
\cite{potvin-2016-google}.
Of course, not all software systems in existence today are quite comparable in 
scale, but it serves to contextualize the challenges that developers may face in
understanding the systems they wrangle each day.
The size, complexity, and indirection that is becoming commonplace in systems
today \cite{latoza-2010-reach} are pushing the already-elusive goal of
enabling developers to more easily enderstand their code into new and 
uncharted territories.

\section{Questions that Developers Ask}
\label{sec:QuestionsThatDeveloperAsk}

\noindent Developers have a large number of information needs that must be
addressed during their day-to-day workflows.
These information needs often present themselves as a variety of questions
that require the integration of often-disparate sources of information
\cite{fritz-2010-info-frag}.
For example, to understand how a defect was introduced, a developer may have to
read the source code of a program, find additional context within a bug report
or a project tracker, or (in the worst case) resort to speaking to other 
developers who may have more information. 

\par Other questions that developers may ask are more focused, and may involve
querying only one source of information, such as a single program or a
collection of modules in a system.
As developers work to evolve and understand specific parts of their systems,
they may ask themselves: ``where are instances of this class created?" or
``what data can we access from this object?" as they try to discover entities
and relationships that capture incoming connections to a given entity
\cite{sillito-2006-questions-during-task}.
Additional questions that developers may ask may be related to 
the control-flow or the data-flow through a program, such as 
``why isn't control reaching this point in code?" and 
``what parts of this data structure are being accessed in this code?" 
\cite{sillito-2006-questions-during-task}.
These are questions that commonly arise as developers perform change tasks
on their code.

\par Researchers have conducted studies that investigate both the 
frequency and difficulty of these focused questions that developers ask during 
change tasks.
LaToza et al. found that a significant portion of a developer's work involves 
answering \emph{reachability questions}, which are defined as a search across 
feasible paths through a program for target statements matching search criteria 
\cite{latoza-2010-reach}.
In a survey distributed to 2000 developers employed at Microsoft's Redmond
campus, 460 developers reported asking questions that could be phrased as 
reachability questions more than 9 times a day.
Of the developers who responded, 82\% of them rated one or more of the
reachability questions as at least somewhat hard to answer.
\cite{latoza-2010-reach}.
Not only are reachability questions difficult to answer, but they are also
time-consuming.
A field study of 17 developers found that of the 10 longest activities
undertaken by devevlopers, 9 were associated with reachability questions
\cite{latoza-2010-reach}.

% TODO: does feel a little weird to end here; should tie it up, somehow.

\section{Tools that Support Developer Questions}
\label{sec:ToolsSupportDeveloperQuestions}

\noindent Different approaches have been taken to building tools 
that enable developers to answer questions about their code.
Some tools such as \emph{Whyline} present a more passive approach,
presenting developers with a subset of pre-defined questions and answers 
they may choose to investigate after runtime \cite{ko-2004-whyline}.
Other tools present a more active approach; the \emph{GetMeHere} tool developed
at Microsoft Research enables developers to form their own queries about
the behaviour of their programs, and provides answers to them without needing
an explicit execution trace \cite{barnett-2014-get}.

\par The Whyline was initially developed as an interogative debugging interface
for the Alice programming language, and later augmented to support the Java
programming language and the needs of professional developers.
At a high-level, Whyline enaables developers to select questions about a
program's output, and it then helps developers to work backfrom from output
to its causes \cite{ko-2009-java-whyline}.
These questions about program output are phrased as ``why did" and ``why didn't"
queries about the values in a program. 
For example, consider a case when a variable \texttt{foo} is being erroneously 
set to some value \textit{x}.
A developer could choose to investigate the cause of this erroneous program
behaviour by running the program, opening the Whyline window and selecting
``why did \texttt{foo} $=$ \textit{x}?" and a \emph{temporal context}, which is 
 a specified time-frame that they wish to investigate.
The Whyline then presents a view of all statments of the program within the
temporal context that contributed to the line where \texttt{foo} has the value
\textit{x}. 
Whyline's approach of presenting a pre-defined subset of questions about the 
output of programs is beneficial, as it avoid speculation about the causes of
a failure, and simplifies the exploration of code responsible for the output
\cite{ko-2009-java-whyline}.
An experiment was conducted to evaluate the efficacy of Whyline on isolating the
causes of two bug reports from an open source projects.
It was found that Whyline users were successful about three times as often
and about twice as fast compared to the control group who used conventional
debugging tools and techniques.

\par Other tools encourage the developer to take a more active approach in
formulating questions about programs and their behaviour.
Similar to Whyline, GetMeHere enables developers to select a line of code as a
point of interest and visualize the execution that reaches the given point.
Where it differs, however, is in how it is able to generate this execution
by using an SMT-based static analysis \cite{barnett-2014-get} without having to
execute the code.
GetMeHere is also presented as a general query engine, where developers can
specify a slice in a program, and execute queries that represent reachability
questions over the given slice \cite{barnett-2014-get}.
The flexibility afforded by GetMeHere enables to developers to fine-tune their
queries, but may introduce some of the speculation about program failures or
erroneous behaviour that Whyline attempts to minimize.

% TODO: talk a bit about performance?
% Wasn't tested on a production system
% Calculator: 650 lines of C#
% Tetris: 3000 lines of C#
% Tailspin Toys: 11k lines of C#

\endinput

Sillito paper shows that developers often select the incorrect tool/are not
aware of the tools they should use to find out answers to these questions.

