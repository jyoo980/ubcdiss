%% The following is a directive for TeXShop to indicate the main file
%%!TEX root = diss.tex

\chapter{Related Work}
\label{ch:RelatedWork}

\begin{epigraph}
    \emph{
       The computing scientist’s main challenge is not to get confused by the 
       complexities of his own making.
     } ---~Edsger W. Dijkstra
\end{epigraph}

\noindent Researchers have investigated both the questions that developers ask 
as they understand and evolve their systems, as well as the efficacy of tools 
that support developers in answering their questions during software change 
tasks.

\par Developers find it challenging to understand code in modern codebases for
a number of reasons.
Many codebases today are large and complex, with some meeting the definition of 
an \emph{ultra-large-scale system} \cite{feiler-2006-ulss}.

\par Google reported in 2016 that its monolithic software codebase was composed
of approximately 1 billion files, with a history of 35 million commits in a
repository that contains 85 terabytes of data and 2 billion lines of code
\cite{potvin-2016-google}.
Of course, not all software systems in existence today are quite comparable in 
scale, but it serves to contextualize the challenges that developers may face in
understanding the systems they wrangle each day.
The size, complexity, and indirection that is becoming commonplace in systems
today \cite{latoza-2010-reach} are pushing the already-elusive goal of
enabling developers to more easily enderstand their code into new and 
uncharted territories. However, this is not an impossible goal.

\section{Questions that Developers Ask}
\label{sec:QuestionsThatDeveloperAsk}

\noindent Developers have a large number of information needs that must be
addressed during their day-to-day workflows.
These information needs often present themselves as a variety of questions
that require the integration of often-disparate sources of information
\cite{fritz-2010-info-frag}.
For example, to understand how a defect was introduced, a developer may have to
read the source code of a program, find additional context within a bug report
or a project tracker, or resort to speaking to other developers who may have 
more information. 

\par Other questions that developers may ask are more focused, and may involve
querying only one source of information, such as a single program or a
collection of modules in a system.
As developers work to evolve and understand specific parts of their systems,
they may ask themselves: "where are instances of this class created?" or
"what data can we access from this object?" as they try to discover entities
and relationships that capture incoming connections to a given entity
\cite{sillito-2006-questions-during-task}.
Additional questions that developers may ask could be related to 
the control-flow or the data-flow through a program, such as 
"why isn't control reaching this point in code?" and 
"what parts of this data structure are being accessed in this code?" 
\cite{sillito-2006-questions-during-task}.

\endinput

